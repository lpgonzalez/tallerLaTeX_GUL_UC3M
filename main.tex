\documentclass[10pt, letterpaper, twoside]{book}

% Comentar si se usa el compilador LuaLaTeX. Descomentar si se usa pdfLaTeX
% \usepackage[utf8]{inputenc}

% Especificar uso de idiomas en el documento, para elementos como índices de contenido
\usepackage[english, spanish, es-tabla]{babel}

% Definición y uso de colores para elementos
\usepackage{xcolor}

% Inserción de imágenes raster
\usepackage{graphicx}

% Inserción de gráficos vectoriales SVG
\usepackage{svg}

% Para poder incluir \blacksquare
\usepackage{amssymb}

% Para poder aplicar enlaces dentro del documento
\usepackage[hidelinks]{hyperref}

% Para tratar adecuadamente los capítulos en el índice de contenido
\usepackage{titlesec}
\usepackage{titletoc}

% Para poder utilizar otras fuentes (requiere un compilador como LuaLaTeX)
\usepackage{fontspec}

% Para trabajar fácilmente con emojis (requiere un compilador como LuaLaTeX)
\usepackage{emoji}

% Para mostrar adecuadamente la tabla
\usepackage{multirow}

% Para gestión de la bibliografía
\usepackage{csquotes}
\usepackage{biblatex}
\addbibresource{bibliografía.bib} % fichero con la bibliografía

% Para gestión de acrónimos y glosario
\usepackage[acronym,toc]{glossaries}
\makeglossaries
\loadglsentries{glosario}

% Meta-información con entradas de acrónimo y glosario
\title{Primer documento}
\author{Lisardo Prieto \thanks{gracias al \gls{GULa} - \gls{UC3Ma}}}
\date{Diciembre 2021}

% Definición de colores personalizados
\definecolor{azulUC3M}{RGB}{24, 30, 146}
\definecolor{amarilloFosforito}{HTML}{FFFF00}

\begin{document}

% Tipo de numeración de página. Números romanos
\pagenumbering{Roman}

% Título del documento usando la meta-información
\maketitle

% -----------------------------------------
% HOLA, ESTO QUE ESCRIBO NO VA A MOSTRARSE
% ESTO ES OTRO COMENTARIO
% -----------------------------------------

% Índices de contenido
\tableofcontents
\listoffigures
\listoftables

% Contenido principal, capítulos, secciones y sub-secciones
\pagenumbering{arabic}

\chapter{Elementos básicos}
\label{chap:básicos} % Etiqueta para enlazar al capítulo desde el texto

\section{Texto}

\subsection{Texto simple}

Mi primer documento en \LaTeX. Ejemplo simple
sin paquetes adicionales.

% -------------------------

\subsubsection{Formatos para texto simple}

Lo más \textbf{importante}
de este \underline{documento} 
no es la \textit{cursiva}.
También podemos \textbf{\textit{encadenar formatos}}.

% -------------------------

\subsection{Colores en el texto}

Hola, soy un \textcolor{red}{texto en rojo} y yo un \textcolor{blue}{texto en azul} y en \textcolor{green}{verde}.

% -------------------------

\section{Texto con color personalizado y subrayado de color}

Ahora podemos poner \textcolor{azulUC3M}{UC3M} en color corporativo y \colorbox{amarilloFosforito}{subrayar texto} en amarillo ``fosforito’’.

% -------------------------

\section{Salto de página}

% Salto de página
\newpage

% -------------------------

\section{Salto de línea}

Texto
% Salto de línea
\newline
Otro texto en nueva línea

% -------------------------

\section{Espaciado personalizado entre líneas de texto}

Hola, soy un texto

% Espaciado personalizado
\vspace{1cm}
Y yo soy otro texto con una separación de 1 cm.

\vspace{50pt}
Y yo soy otro texto más alejado.

\section{Listas simples}

Las listas son sencillas. Para crearlas:
\begin{itemize}
  \item Las entradas de la lista comienzan con el comando \verb|\item| .
  \item Este tipo de lista tiene cada entrada empezando con un ``bullet''.
  \item La longitud de cada entrada es arbitraria.
  \item Y el número de elementos también lo es.
\end{itemize}

% -------------------------

\section{Listas enumeradas}

Las listas enumeradas también son sencillas:
\begin{enumerate}
  \item Los elementos se enumeran automáticamente.
  \item Los números  se crean automáticamente en el entorno \texttt{enumerate}.
  \item Otra entrada
\end{enumerate}

% -------------------------

\subsection{Sublistas enumeradas}

Ejemplo de sublistas enumeradas:
\begin{enumerate}
  \item Primer nivel.
  \begin{enumerate}
    \item Segundo nivel.
    \begin{enumerate}
      \item Tercer nivel.
      \item Tercer nivel.
    \end{enumerate}
    \item Segundo nivel.
  \end{enumerate}
  \item Primer nivel.
\end{enumerate}

% -------------------------

\section{Más sublistas}

Otro ejemplo de sublistas:
\begin{itemize}
  \item Primer nivel.
  \begin{itemize}
    \item Segundo nivel.
    \begin{itemize}
      \item Tercer nivel.
      \item Tercer nivel.
    \end{itemize}
    \item Segundo nivel.
  \end{itemize}
  \item Primer nivel.
\end{itemize}

% -------------------------

\section{Listas personalizadas}

Tanto las listas con puntos como las enumeradas permiten cambiar el elemento de inicio con \verb|\item[elemento]|
\begin{enumerate}
  \item Primer punto
  \item Segundo punto
  \item[!] Tercer punto, comenzando con exclamación
  \item[$\blacksquare$] Ahora empezando con un cuadrado negro
  \item[NOTA] Esta entrada no tiene ``bullet‘’
  \item[] Entrada en blanco?
\end{enumerate}


\chapter{Elementos avanzados}
\label{chap:avanzados} % Etiqueta para enlazar al capítulo desde el texto

\section{Imágenes}

\subsection{Raster}

\begin{figure}[ht]
    \centering
    \includegraphics[height=7cm]{recursos/gul_logo_raster.png}
    \caption{Logo del GUL en formato PNG}
    \label{fig:logo1}
\end{figure}

La figura \ref{fig:logo1} se encuentra en la página \pageref{fig:logo1}.

\newpage

% -------------------------

\subsection{Vectorial (svg)}

\begin{figure}[ht]
     \includesvg[width=\textwidth]{recursos/gul_logo_vector.svg}
     \centering
     \caption{Logo del GUL en formato vectorial (svg)}
     \label{fig:logo2}
\end{figure}

\section{Tablas}

\begin{table}[ht]
  \begin{tabular}{ |c|c|c| } 
    \hline
    Columna A & Columna B & Columna C\\ 
    1 & 3 & 5 \\ 
    2 & 4 & 6 \\ 
    \hline
  \end{tabular}
  \centering
  \caption{Tabla de ejemplo}
  \label{tab:tabla1}
\end{table}

\vspace{1cm}

\begin{table}[ht]
\begin{tabular}{ccccc}
\cline{1-4}
\multicolumn{2}{|c|}{emosido}                       & \multicolumn{1}{c|}{}      & \multicolumn{1}{c|}{}     &                       \\ \cline{1-4}
\multicolumn{1}{|c|}{} & \multicolumn{2}{c|}{\multirow{2}{*}{\textbf{engañado}}} & \multicolumn{1}{c|}{}     &                       \\ \cline{1-1} \cline{4-4}
\multicolumn{1}{|c|}{} & \multicolumn{2}{c|}{}                                   & \multicolumn{1}{c|}{}     &                       \\ \cline{1-4}
\multicolumn{1}{|c|}{} & \multicolumn{1}{c|}{}      & \multicolumn{1}{c|}{}      & \multicolumn{1}{c|}{hola} &                       \\ \cline{1-4}
                       &                            &                            &                           & \textit{Hello there!}
\end{tabular}
\centering
\caption{Otra tabla de ejemplo}
\label{tab:tablaEmosido}
\end{table}
\section{Emojis}

% -------------------------

\subsection{Emojis con el paquete ``emoji''}

Emojis con colores gracias al paquete \texttt{emoji} y LuaLaTeX:
\emoji{leaves}
\emoji{rose}

Otro ejemplo, en blanco y negro, usando el código Unicode para carita feliz (U+1F600): {\fontspec{Symbola}\symbol{"1F600}}

% -------------------------

\subsection{Emojis con el tipografía específica (NotoColorEmoji.ttf) y códigos de caracter Unicode}

\newfontfamily\emojifont[Renderer=Harfbuzz]{NotoColorEmoji.ttf}

Emojis utilizando una tipografía específica y código de caracter Unicode (U+1F600):
% Utilizar \emojifont en un grupo para hacer que su efecto sea local
{\emojifont \Uchar"1F600 \char"1F600}
% http://www.unicode.org/emoji/charts/full-emoji-list.html

\section{Acrónimos y glosario}

El \gls{GULa} (o \gls{GULg}) forma parte de la \gls{UC3Ma} (\gls{UC3Mg}).

\section{Bibliografía y citas}

Cita a una referencia \cite{gonzalez2020simultaneous}.



% Acrónimos y glosario de términos
\printglossary[type=\acronymtype,title=Acrónimos]
\printglossary[type=main]

% Bibliografía
\printbibliography

\end{document}