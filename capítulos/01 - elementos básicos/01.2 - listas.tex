\section{Listas simples}

Las listas son sencillas. Para crearlas:
\begin{itemize}
  \item Las entradas de la lista comienzan con el comando \verb|\item| .
  \item Este tipo de lista tiene cada entrada empezando con un ``bullet''.
  \item La longitud de cada entrada es arbitraria.
  \item Y el número de elementos también lo es.
\end{itemize}

% -------------------------

\section{Listas enumeradas}

Las listas enumeradas también son sencillas:
\begin{enumerate}
  \item Los elementos se enumeran automáticamente.
  \item Los números  se crean automáticamente en el entorno \texttt{enumerate}.
  \item Otra entrada
\end{enumerate}

% -------------------------

\subsection{Sublistas enumeradas}

Ejemplo de sublistas enumeradas:
\begin{enumerate}
  \item Primer nivel.
  \begin{enumerate}
    \item Segundo nivel.
    \begin{enumerate}
      \item Tercer nivel.
      \item Tercer nivel.
    \end{enumerate}
    \item Segundo nivel.
  \end{enumerate}
  \item Primer nivel.
\end{enumerate}

% -------------------------

\section{Más sublistas}

Otro ejemplo de sublistas:
\begin{itemize}
  \item Primer nivel.
  \begin{itemize}
    \item Segundo nivel.
    \begin{itemize}
      \item Tercer nivel.
      \item Tercer nivel.
    \end{itemize}
    \item Segundo nivel.
  \end{itemize}
  \item Primer nivel.
\end{itemize}

% -------------------------

\section{Listas personalizadas}

Tanto las listas con puntos como las enumeradas permiten cambiar el elemento de inicio con \verb|\item[elemento]|
\begin{enumerate}
  \item Primer punto
  \item Segundo punto
  \item[!] Tercer punto, comenzando con exclamación
  \item[$\blacksquare$] Ahora empezando con un cuadrado negro
  \item[NOTA] Esta entrada no tiene ``bullet‘’
  \item[] Entrada en blanco?
\end{enumerate}
